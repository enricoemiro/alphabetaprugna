\section{Soluzione adottata}
Per risolvere il problema sopra citato, il "giocatore" è stato implementato utilizzando "alpha-beta pruning" come algoritmo di ricerca.\\
La profondità massima dell'alpha-beta viene gestita applicando la strategia di "iterative deepening".\\
Le configurazioni di gioco sono valutate attraverso un'euristica basata sul numero di celle consecutive di ogni giocatore.\\
Le celle che vengono esplorate sono ordinate dando priorità a quelle più vicine all'ultima mossa e con una posizione migliore.\\
Infine, per evitare di valutare le stesse configurazioni di gioco più volte, esse vengono salvate all'interno di una hash table nota come "tabella delle trasposizioni".

\subsection{Select Cell}
Il metodo "Select Cell" si occupa di scegliere la mossa che il giocatore deve eseguire. Esso, nel caso in cui il nostro giocatore sia il primo a giocare, sceglie direttamente la cella centrale della matrice di gioco, poichè permette di avere il maggior numero di possibilità di movimento.\\
Se invece non siamo i primi a giocare, la scelta della cella è affidata all'algoritmo Alpha-beta pruning.\\
Nel caso in cui l'Alpha-beta non riesca a calcolare la mossa entro il tempo limite, viene scelta una mossa casuale tra quelle disponibili.

\subsection{Alpha-beta pruning}
Alpha-beta pruning è una variante dell'algoritmo minimax, che riduce drasticamente il numero di nodi da valutare nell'albero di ricerca e si basa su due valori: $ \alpha $ e $ \beta $.\\
$ \alpha $ e $ \beta $ rappresentano, in ogni punto dell'albero, la posizione migliore e peggiore che è possibile raggiungere a partire dalla posizione $ (i, j) $.\\
Se A è il giocatore massimizzante, B è il giocatore minimizzante e $ (i, j) $ è la posizione di partenza:
\begin{itemize}
    \item $ \alpha $: è il punteggio minimo che A può raggiungere a partire da $ (i, j) $.
    \item $ \beta $: è il punteggio massimo che B può raggiungere a partire da $ (i, j) $.
\end{itemize}
Se $ \beta \leq \alpha $, allora il sottoalbero relativo al nodo verrà "potato" perchè sicuramente la soluzione non sarà ottima.
